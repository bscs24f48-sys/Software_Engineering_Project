\documentclass[12pt,a4paper]{article}

% Packages
\usepackage[utf8]{inputenc}
\usepackage[margin=1in]{geometry}
\usepackage{graphicx}
\usepackage{hyperref}
\usepackage{titlesec}
\usepackage{fancyhdr}
\usepackage{parskip}
\usepackage{enumitem}
\usepackage{tocloft}
\usepackage{setspace}
\usepackage{cite}
\usepackage{xcolor}
\renewcommand{\contentsname}{\color{black}Table of Contents}

% Header and Footer
\pagestyle{fancy}
\fancyhf{}
\fancyhead[L]{CSC-225 Software Engineering}
\fancyhead[R]{Milestone 1}
\fancyfoot[C]{\thepage}

% Title formatting
\titleformat{\section}{\Large\bfseries}{\thesection}{1em}{}
\titleformat{\subsection}{\large\bfseries}{\thesubsection}{1em}{}

% Hyperlink setup
\hypersetup{
    colorlinks=true,
    linkcolor=blue,
    filecolor=magenta,      
    urlcolor=cyan,
    citecolor=blue,
}
\hypersetup{
    colorlinks=true,
    linkcolor=black,
    linktoc=all
}

\onehalfspacing

\begin{document}

% ========================
% TITLE PAGE
% ========================
\begin{titlepage}
    \centering
    \vspace*{2cm}
    
    {\Huge\bfseries "Multi-Channel Advertising Management Platform"\par}
    {\Huge\bfseries Adlynk\par}
    \vspace{1cm}
    {\Large Software Engineering Project Proposal\par}
    \vspace{0.5cm}
    {\large CSC-225 – Software Engineering\par}
    \vspace{2cm}
    
    {\Large\textbf{Submitted By:}\par}
    \vspace{0.5cm}
    \begin{center}
\begin{tabular}{llp{4cm}} % Two 'l' for left alignment, one 'p' for fixed width for details
\textbf{Student Name 1} & \textbf{Bilal Ahmed Khan} \\  
\textbf{Roll NO:} &  \textbf{NUM-BSCS-2024-21}\\
 \textbf{EMAIL:} &     \textbf{bscs24f21@namal.edu.pk}  \\[0.3cm] \\
 \textbf{Student Name 2} & \textbf{Muhammad Hamza} \\  
\textbf{Roll NO:} &  \textbf{NUM-BSCS-2024-48}\\
 \textbf{EMAIL:} &     \textbf{bscs24f48@namal.edu.pk}  \\[0.3cm]



\end{tabular}
\end{center}
    
    \vfill
    
    {\large\textbf{Submission Date:} November 9, 2025\par}
    \vspace{0.5cm}
    {\large Department of Computer Science\par}
    {\large Namal University, Mianwali\par}
    
\end{titlepage}
\newpage
\tableofcontents
\newpage

% ========================
% REQUIREMENT PROVIDER AGREEMENT
% ========================
\newpage
\section{Parties Involved}

\subsection*{Requirement Provider (RP):}
\begin{itemize}
    \item Name: Saqlain Baig
    \item Role: Client/Requirement Provider
    \item Contact: 03336308826
    \item Email: saqlainbaig5@gmail.com
\end{itemize}

\subsection*{Development Team:}
\textbf{Developer 1:}
\begin{itemize}
    \item Name: Muhammad Hamza
    \item Roll No: NUM-BSCS-2024-48
    \item Email: bscs24f48@namal.edu.pk
\end{itemize}

\textbf{Developer 2:}
\begin{itemize}
    \item Name: Bilal Ahmed Khan
    \item Roll No: NUM-BSCS-2024-21
    \item Email: bscs24f21@namal.edu.pk
\end{itemize}

\noindent\textbf{Academic Institution:} Namal University, Mianwali

\noindent\textbf{Course:} CSC-225 -- Software Engineering

\section{Purpose of Agreement}

This agreement establishes the terms and understanding between the Requirement Provider (RP) and the development team regarding the software engineering project titled ``Multi-Channel Advertising Management Platform.'' The RP agrees to provide necessary requirements, feedback, and guidance throughout the project lifecycle.

\section{Project Overview}

The project aims to develop a Multi-Channel Advertising Management Platform that centralizes advertising services including billboards, vehicle branding, and social media marketing into a single digital platform. The system will enable business owners to easily find, book, and manage advertising campaigns while connecting them with professional advertisers, billboard owners, and social media influencers.

\section{Responsibilities of the Requirement Provider}

The Requirement Provider, Saqlain Baig, agrees to:

\begin{enumerate}
    \item \textbf{Provide Clear Requirements:} Share detailed information about the desired features, functionality, and objectives of the platform.
    
    \item \textbf{Availability for Communication:} Be available for regular meetings (weekly or bi-weekly) to discuss project progress, provide feedback, and clarify requirements.
    
    \item \textbf{Timely Feedback:} Review deliverables and prototypes within an agreed timeframe (typically 3-5 business days) and provide constructive feedback.
    
    \item \textbf{Domain Expertise:} Share industry knowledge about advertising management, business workflows, and user needs to ensure the platform meets real-world requirements.
    
    \item \textbf{Testing and Validation:} Participate in user acceptance testing (UAT) and validate that the developed features meet the specified requirements.
    
    \item \textbf{Access to Resources:} Provide any necessary resources, sample data, or access to relevant stakeholders that may help in understanding the requirements better.
\end{enumerate}

\section{Responsibilities of the Development Team}

The development team (Muhammad Hamza and Bilal Ahmed Khan) agrees to:

\begin{enumerate}
    \item \textbf{Requirement Analysis:} Conduct thorough analysis of the requirements provided by the RP and document them clearly.
    
    \item \textbf{Regular Updates:} Provide regular progress updates through meetings, demos, and milestone reports.
    
    \item \textbf{Quality Deliverables:} Develop the platform following software engineering best practices and deliver functional, tested features.
    
    \item \textbf{Documentation:} Maintain comprehensive documentation including requirement specifications, design documents, and user manuals.
    
    \item \textbf{Iterative Development:} Follow Agile methodology allowing for feedback and adjustments.
    
    \item \textbf{Problem Solving:} Address technical challenges and seek clarification from the RP when requirements are unclear or need modification.
\end{enumerate}

\section{Project Timeline and Milestones}

The project will be completed in phases over the course of the semester:

\begin{itemize}
    \item \textbf{Milestone 1:} Requirement gathering and project proposal (November 2025)
    \item \textbf{Milestone 2:} System design and architecture (December 2025)
    \item \textbf{Milestone 3:} Development and implementation (January-February 2026)
    \item \textbf{Milestone 4:} Testing and deployment (March 2026)
    \item \textbf{Final Presentation:} End of semester
\end{itemize}

\noindent\textit{Note: Specific dates will be confirmed in project planning meetings.}

\section{Communication Protocol}

\begin{enumerate}
    \item \textbf{Meeting Schedule:} Weekly/bi-weekly meetings (to be scheduled mutually)
    \item \textbf{Communication Channels:}
    \begin{itemize}
        \item Email for formal communication
        \item WhatsApp/Phone for quick queries
        \item Video calls (Zoom/Google Meet) for detailed discussions
    \end{itemize}
    \item \textbf{Response Time:} Both parties agree to respond to queries within 48 hours
\end{enumerate}

\section{Scope and Limitations}

\begin{enumerate}
    \item This is an academic project developed as part of the Software Engineering course (CSC-225).
    \item The final deliverable will be a working prototype/minimum viable product (MVP) demonstrating core functionalities.
    \item Advanced features requiring extensive resources or third-party integrations may be documented as future enhancements.
    \item The project timeline and scope are subject to academic constraints and course requirements.
\end{enumerate}

\section{Intellectual Property and Confidentiality}

\begin{enumerate}
    \item All project requirements, business information, and data shared by the RP will be kept confidential and used solely for this academic project.
    \item The developed software and documentation will be owned by the development team for academic purposes, with the RP having rights to use the concept and requirements for future development.
    \item Both parties agree not to share sensitive project information with unauthorized third parties.
\end{enumerate}

\section{Modification of Requirements}

\begin{enumerate}
    \item Any changes to the project requirements must be discussed and mutually agreed upon by both parties.
    \item Major changes in scope may require timeline adjustments and will be documented through change request forms.
    \item The development team reserves the right to assess the feasibility of requested changes and provide alternative solutions if necessary.
\end{enumerate}

\section{Academic Compliance}

This project is conducted as part of academic coursework at Namal University. Both parties acknowledge that:

\begin{enumerate}
    \item The project must meet academic standards and course requirements.
    \item The supervising faculty has the authority to provide guidance and make decisions regarding project scope and direction.
    \item All work must adhere to academic integrity policies.
\end{enumerate}

\section{Dispute Resolution}

In case of any disagreements or conflicts:

\begin{enumerate}
    \item Issues will first be discussed directly between the RP and development team.
    \item If unresolved, the course instructor will be consulted for mediation.
    \item Both parties agree to work collaboratively and professionally to resolve disputes.
\end{enumerate}

\section{Agreement Termination}

This agreement remains valid for the duration of the academic semester. Either party may request termination with valid reasons, subject to approval by the course instructor.

\section{Signatures}

By signing below, both parties acknowledge that they have read, understood, and agree to the terms outlined in this Requirement Provider Agreement.

\vspace{1cm}

\noindent\textbf{Requirement Provider:}

\noindent\textbf{Name:} Saqlain Baig

\noindent\textbf{Signature:} \underline{\hspace{6cm}}

\noindent\textbf{Date:} \underline{\hspace{6cm}}

\vspace{1cm}

\noindent\textbf{Development Team Member 1:}

\noindent\textbf{Name:} Muhammad Hamza

\noindent\textbf{Signature:} \underline{\hspace{6cm}}

\noindent\textbf{Date:} \underline{\hspace{6cm}}

\vspace{1cm}

\noindent\textbf{Development Team Member 2:}

\noindent\textbf{Name:} Bilal Ahmed Khan

\noindent\textbf{Signature:} \underline{\hspace{6cm}}

\noindent\textbf{Date:} \underline{\hspace{6cm}}

\vspace{1cm}

\noindent\textbf{Course Instructor (Witness):}

\noindent\textbf{Name:} \underline{\hspace{6cm}}

\noindent\textbf{Signature:} \underline{\hspace{6cm}}

\noindent\textbf{Date:} \underline{\hspace{6cm}}

\section{Appendices}

\noindent\textbf{Appendix A:} Contact Information Summary

\noindent\textbf{Appendix B:} Project Milestone Schedule (to be attached)

\noindent\textbf{Appendix C:} Requirement Change Request Form (if applicable)


% ========================
% REQUIREMENT PROVIDER AGREEMENT
% ========================

\vspace{1cm}




\vspace{1cm}
\section{Introduction}
Advertising is a crucial part of every business, but managing campaigns across multiple platforms can be complex and time-consuming. Our \textbf{Multi-Channel Advertising Management Platform} provides a \textbf{digital solution} that brings all advertising options together in one place — from \textbf{online promotions and social media marketing} to \textbf{traditional methods like billboards and vehicle branding}.

Business owners no longer need to contact multiple agencies or negotiate separate deals. Through our platform, they can easily find and book advertising services digitally, as well as reserve billboard spaces in different cities across Pakistan — all from a single, convenient dashboard.

This solution operates under \textbf{ADLynk}, which leverages digital innovation to make advertising easier and more effective. It makes professional advertising accessible to everyone — not just large corporations and retailers, but also \textbf{small businesses, startups, and online store owners} — enabling them to run complete marketing campaigns with transparent pricing, simplified booking, and clear performance tracking.

By centralizing the entire advertising process, our platform saves time, reduces costs, and improves coordination across all marketing channels, while also helping \textbf{advertisers, billboard owners, and influencers} connect with more clients efficiently.


% ========================
% PROBLEM STATEMENT
% ========================
\section{Problem Statement}

In today’s competitive market, businesses rely on multiple advertising channels to reach their target audience — including billboards, vehicle branding, and social media platforms. However, managing these campaigns across different platforms is \textbf{complicated, time-consuming, and inefficient}.

Currently, \textbf{business owners must contact multiple agencies} and \textbf{negotiate separate deals}, which leads to \textbf{increased costs, inconsistent messaging, and poor time management}. \textbf{Small businesses, startups, and online store owners} often struggle even more due to limited resources and lack of access to professional advertisers and advertising networks.

There is \textbf{no unified digital system in Pakistan} that connects business owners with \textbf{professional advertisers, billboard owners, vehicle advertisers, and social media influencers} under one platform. As a result, both \textbf{business owners and service providers} face challenges — business owners find it hard to plan and manage multi-channel campaigns effectively, while service providers miss potential clients due to limited visibility.

 

\textbf{Current Problem:}

In the current advertising landscape, \textbf{business owners in Pakistan face major challenges} when trying to promote their brands across different platforms. To run a complete marketing campaign, they have to \textbf{individually contact billboard owners, vehicle advertisers, and digital marketing agencies or social media influencers}, which makes the entire process \textbf{scattered, slow, and inefficient}.

Most \textbf{traditional advertising methods} still rely on manual coordination, personal visits, and offline negotiations. On the other hand, \textbf{digital advertising} through social media platforms often lacks transparency in pricing, performance tracking, and trust between business owners and advertisers.

Because there is \textbf{no centralized digital system} that brings all these services together, business owners spend extra \textbf{time, effort, and money} managing different campaigns separately. As a result, they struggle to maintain \textbf{consistent branding}, measure \textbf{marketing effectiveness}, and make \textbf{data-driven decisions}.

This fragmented approach prevents small and medium-sized businesses from running effective, multi-channel campaigns, while advertisers and service providers also lose potential clients due to lack of visibility and accessibility.


\textbf{Who is Affected:}

The challenges in the current advertising system affect multiple stakeholders within the marketing ecosystem: 
    \begin{itemize}
        \item \textbf{Business Owners:} Especially small businesses, startups, and online store owners who lack the resources or connections to manage multiple advertising channels. They struggle to find reliable advertisers, compare prices, and coordinate campaigns efficiently.
    \end{itemize}
    \begin{itemize}
        \item \textbf{Advertisers, Influencers and Marketing Agencies:} Many professional advertisers, Influencers and digital marketers lose potential clients because there is no centralized platform where businesses can easily discover and connect with them.
    \end{itemize}

               
\begin{itemize}
    \item \textbf{Billboard and Vehicle Advertising Providers:}Owners of billboard spaces and advertising vehicles face difficulties in reaching potential customers and managing bookings due to the absence of a unified digital system.
\end{itemize}
             
            
\textbf{Consequences of Not Solving:} 

If current problems in the advertising industry remain unaddressed, several negative consequences will continue to affect businesses, advertisers, and the overall marketing ecosystem.

\begin{enumerate}
\textbf{    \item Wasted Time and Resources:}
Business owners will keep spending excessive time searching for reliable advertisers, waiting for responses, and dealing with the inconvenience of coordinating with several agencies separately.
\textbf{\item Higher Advertising Costs:Limited Access for Small Businesses:}
Without a centralized system, businesses will continue to face inconsistent pricing, hidden charges, and unstandardized rates, increasing overall marketing expenses.
\textbf{\item Limited Access for Small Businesses:}
Startups and small business owners will remain at a disadvantage, as they lack the resources to connect with professional advertisers, Influencers and plan multi-channel campaigns. Many small businesses lose money by investing in expensive influencers and inexperienced marketers, yet fail to achieve meaningful sales. As a result, they often run out of funds when their business actually needs financial support for real growth. 
\textbf{\item Missed Growth Opportunities for Service Providers:}
Billboard owners, Advertisers and influencers will keep losing potential clients due to limited visibility and lack of digital presence.
\end{enumerate}
\textbf{Inadequacy of Existing Solutions:}

Currently, businesses in Pakistan rely on a fragmented approach to advertising, using multiple agencies, and platforms to reach their target audience. Existing solutions have several limitations: 
\begin{enumerate}
\textbf{    \item Fragmented Platforms:}
There is no single platform that integrates digital advertising, social media influencers, billboards, and vehicle advertising. Businesses must contact multiple agencies individually, which is inefficient and time-consuming. 
\textbf{\item Lack of Transparency:}
Pricing, availability, and performance metrics are often unclear or inconsistent across different service providers, making it difficult for business owners to make informed decisions. 
\textbf{\item Limited Access for Small Businesses:}
Most advertising solutions cater primarily to large corporations. Small businesses, startups, and online stores often cannot access professional advertisers or billboard networks easily.
\textbf{\item \textbf{Manual Coordination and Delays} :}
Business owners must wait for responses from multiple agencies or service providers, resulting in slow campaign execution and poor coordination between different advertising channels. 

\end{enumerate}
% ========================
% PROJECT OBJECTIVES
% ========================
\section{Project Objectives}

The main objectives of the project are to address the challenges faced by business owners and service providers in the advertising industry and provide a unified, efficient, and digital solution. Specifically, the objectives are: 
\textbf{    \item Centralize Advertising Services:}
To provide a single platform where business owners can access and manage multiple advertising channels, including \textbf{billboards, vehicle branding, and social media marketing}.
\item \textbf{Simplify Booking and Coordination:}
To reduce the time and effort business owners spend searching for advertisers, waiting for responses, and negotiating separate deals. 
\item \textbf{Increase Accessibility for Small Businesses and Startups:}
To enable small businesses, startups, and e-commerce brands to run professional marketing campaigns with transparent pricing and easy access to professional advertisers. 
\item \textbf{Enhance Visibility for Service Providers:}
To help billboard owners, vehicle advertisers, and social media influencers reach more clients and grow their business through a centralized digital platform. 
\item \textbf{Promote Digital Transformation in Advertising:}
To modernize the advertising industry in Pakistan by integrating traditional and digital advertising methods into one efficient system. 
\textbf{
\section{Stakeholder Identification}
} 
The success of the Multi-Channel Advertising Management Platform depends on understanding and addressing the needs of all stakeholders involved in the advertising ecosystem. The key stakeholders are: 
\begin{itemize}
    \item \textbf{Business Owners:}
Small businesses, startups, e-commerce brands, and large corporations who want to promote their products or services across multiple channels efficiently.
\item \textbf{Professional Advertisers and Marketing Agencies:}
Digital marketing experts and agencies who provide social media marketing, online promotions, and campaign management services. 
\item \textbf{Billboard Owners:}
Individuals or companies owning billboard spaces who want to maximize occupancy and revenue by reaching more clients.
\item \textbf{Social Media Influencers:}
Individuals or content creators who collaborate with businesses for digital promotions and brand campaigns. 
\item \textbf{Platform Administrators/Developers:}
The team responsible for maintaining the platform, ensuring smooth functionality, security, and continuous improvements. 
\end{itemize}
% ========================
% SOFTWARE DEVELOPMENT METHODOLOGY
% ========================
\section{Software Development Methodology}

The development of the \textbf{Multi-Channel Advertising Management Platform} will follow the \textbf{Agile methodology}, specifically using the \textbf{Scrum framework}. Agile is chosen because it allows \textbf{iterative development, flexibility, and continuous stakeholder feedback}, which are essential for building a platform that meets the dynamic needs of business owners and advertisers. 
% ========================
% TOOLS AND TECHNOLOGIES
% ========================
\section{Tools and Technologies}

The development of the \textbf{ADLynk} will use a combination of design, development, and backend tools to ensure a \textbf{user-friendly, scalable, and efficient mobile application}.

\paragraph{\textbf{1. UI/UX Design}}

\begin{itemize}
    \item \textbf{Adobe XD} – For designing interactive and user-friendly app interfaces, wireframes, and prototypes.

\end{itemize}

\paragraph{\textbf{2. Mobile App Development}}

\begin{itemize}
    \item \textbf{Flutter / React Native} – Cross-platform frameworks for building mobile applications compatible with \textbf{Android and iOS}.
    \item \textbf{Dart / JavaScript} – Programming languages used with Flutter or React Native for app development.
\end{itemize}

\paragraph{\textbf{3. Back-End Development}}

\begin{itemize}
    \item \textbf{Node.js / Python (Django / Flask)} – To manage server-side operations, handle business logic, and process requests from the app.
    \item \textbf{RESTful APIs} – To connect the mobile app with the backend and database services.
\end{itemize}

\paragraph{\textbf{4. Database Management}}


    \item \textbf{Firebase Realtime Database / Firestore} – For storing user data, advertiser details, bookings, and campaign information.







\end{document}